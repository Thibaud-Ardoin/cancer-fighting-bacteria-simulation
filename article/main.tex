\documentclass[a4paper]{article}

%% Language and font encodings
\usepackage[english]{babel}
\usepackage[utf8x]{inputenc}
\usepackage[T1]{fontenc}

%% Sets page size and margins
\usepackage[a4paper,top=3cm,bottom=2cm,left=3cm,right=3cm,marginparwidth=1.75cm]{geometry}

%% Useful packages
\usepackage{amsmath}
\usepackage{amssymb}
\usepackage{graphicx}
\usepackage[colorinlistoftodos]{todonotes}
\usepackage[colorlinks=true, allcolors=blue]{hyperref}

\usepackage{amsthm}
\usepackage {stmaryrd}

\usepackage{listings}
\usepackage{color}

\definecolor{dkgreen}{rgb}{0,0.6,0}
\definecolor{gray}{rgb}{0.5,0.5,0.5}
\definecolor{mauve}{rgb}{0.58,0,0.82}

\lstset{language=sql,
  aboveskip=3mm,
  belowskip=3mm,
  showstringspaces=false,
  columns=flexible,
  basicstyle={\small\ttfamily},
  numbers=none,
  numberstyle=\tiny\color{gray},
  keywordstyle=\color{blue},
  commentstyle=\color{dkgreen},
  stringstyle=\color{mauve},
  breaklines=true,
  breakatwhitespace=true,
  tabsize=3
}

\usepackage{tikz}
\usetikzlibrary{automata,positioning}

\theoremstyle{definition}
\newtheorem{Ex}{Exercice}

\title{Initiation to Research}
\author{Ardoin, Lichtle, Riou}

\begin{document}
\maketitle

\section{A mathematical model to understand local dispersion of AHL}

In this part, our goal is to determine the spatial equations that rule the behaviour of bacteria and the interactions between them via AHL and quorum sensing.

\subsection{Presentation}

First, we consider that the experience takes place in a bounded metric space $E$.

In this space, some bacteria evolve. A \emph{bacterium} is considered as a material point in space $E$, and we denote $p_n(t)$ the position of bacterium $n$ at time $t$.

Moreover, bacteria produce a plasmid LuxI (denoted $I$) and a lysis protein (denoted $L$). At any time $t$, we denote $I_n(t)$ (resp. $L_n(t)$) the intracellular concentration of LuxI (resp. lysis protein) of bacterium $n$. Bacteria also produces AHL at regular intervals. We denote $H : E \times \mathbb{R}_+ \to \mathbb{R}_+$ the function of AHL density, such that $H(M,t)$ represents the density of AHL on position $M$ at time $t$.

Finally, bacteria are able to do two important actions: \emph{reproduce} themselves and \emph{self-destruct}. Therefore, the set of bacteria is not constant, thus we denote $N_t$ the set of all bacteria which exists in space $E$ at time $t$.

\subsection{Modelling}

\subsubsection{Moves of bacteria and AHL particles}

We consider that bacteria move following a brownian movement. In terms of visualization, we thus represent them as spheres of constant radius which move randomly. When any bacterium touches a border of the space, it simply follows the same move but in the symmetric direction. For example, if some bacterium touches a vertical bound while its speed is a vector $(\vec{v}_x, \vec{v}_y)$, its new speed will be $(-\vec{v}_x, \vec{v}_y)$.

TODO collisions cells

We suppose that AHL molecules are very small in comparison with bacteria, therefore we represent these molecules by a function of density defined on all the space $E$. This density depends on the ponctual production of AHL by bacteria, but also evolves temporaly because of the propagation of these molecules in the fluid in which they are produced.

We represent this propagation using convolutions, such that at any time $t$ :
\begin{equation}
\frac{dH}{dt}(x,y,z,t) = (H * g)(x,y,z,t) := \iiint \limits_E H(x-u,y-v,z-w,t)g(u,v,w) du dv dw
\end{equation}
where $g$ is a filter defined on $E$.

Using convolutions to represent a propagation of our signal is inspired from classical algorithms of image processing. Intuitively, the idea is to choose a well defined filter $g$ such that the convolution operation "blurs" the density of AHL, which corresponds to the phenomenon of density homogenization over time.

In practice, for computation considerations, the space will be discretized, such that the integrals will become discrete sums, and that the filter $g$ will be represented as a matrix.



\subsubsection{Updates of bacteria}

We consider that cells have an internal clock, and that they "update" themselves every $\Delta t$. At every update, each bacterium $n$ does the following actions:
\begin{itemize}

\item \textbf{Self-destruction:} The self-destruction of a cell is more likely to happen when the intracellular concentration of lysis protein is high. We first define the cell degradation function $\gamma_n$:
$$\gamma_n(t) := \frac{kL_n^2(t)}{L_{n,0}^2 + L_n^2(t)}$$
where $k$ is the maximum rate of cell lysis and $L_{n,0}$ is the concentration of lysis gene resulting in half maximum lysis.

With a probability $\mu_D := \min(\gamma_n(t), 1)$, bacterium $n$ self-destructs.


\item \textbf{Reproduction:} With a probability $\mu_G$, the bacterium divides itself into two identical bacteria $n_1$ and $n_2$. The concentration of lysis protein and LuxI are equally shared by the two new bacteria :
\begin{equation}
L_{n_1}(t+\Delta t) =  L_{n_2}(t+\Delta t) = \frac{L_n(t)}{2}
\end{equation}
\begin{equation}
I_{n_1}(t+\Delta t) =  I_{n_2}(t+\Delta t) = \frac{I_n(t)}{2}
\end{equation}


\item \textbf{AHL production:} At each step, bacteria produce a certain quantity of AHL, proportionnally to the intracellular concentration of LuxI. Mathematically, it can be formalized by the partial derivative equation :
$$H(p_n(t+\Delta_t), t+\Delta t) = H(p_n(t+\Delta t),t) +  bI_n(t)$$

\item \textbf{Update of $L$ and $I$:} Like we saw in previous part, the concentration of lysis protein is directly linked to AHL concentration around the cell.

We consider that the quantity $H_n(t)$ of AHL that can be detected by a bacterium $n$ at time $t$ is :
$$H_n(t) := \iiint \limits_{x \in B(p_n(t), \rho)} H(x, t)$$
where $\rho$ is the maximum distance of AHL detection for a bacterium and $B(x, \rho)$ denotes the ball of center $x \in E$ and radius $\rho$.

The activation term $P_n$ is thus defined by :
$$P_n(t) := \alpha_0 + \frac{\alpha_H(H_n(t)/H_0)^4}{1 + (H_n(t)/H_0)^4}$$
where $\alpha_0$ is the Lux promoter basal detection, $\alpha_H$ is the Lux promoter AHL induced detection and $H_0$ is the AHL binding affinity to Lux promoter.

$L_n$ and $I_n$ then satisfy the following differential equations :
\begin{equation}
\frac{d L_n}{d t} = C_L P_n - \gamma_L L
\end{equation}
\begin{equation}
\frac{d I_n}{d t} = C_I P_n - \gamma_I I
\end{equation}
\end{itemize}

\subsection{Implementation}

In practice, our simulation is realized on a discretized space, and quantity evolve by successive cycles of time $\Delta t$. In order to limit the complexity of the simulation, we also consider that all bacteria have the same internal clock, thus all cells update at the same time.

Our space $E$ is totally discrete, so the function $H$ is represented by a big matrix, in which each coefficient represents the concentration of AHL in an elementary volume of space.

Moreover, the derivative equations are also discrete, such that for all function $f$, we make the following approximation :
$$\frac{df}{dt} = \frac{f(t+\Delta t) - f(t)}{\Delta t}$$



\end{document}